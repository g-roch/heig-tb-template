
\usepackage{heig}

%%%%%%%%%%%%%%%%%%%%%%%%%%%%%%%%%%%%%%%%%%%%%%%%%%%%%%%%%%
% Configuration des répertoires
%
\graphicspath{ {./images} }

%%%%%%%%%%%%%%%%%%%%%%%%%%%%%%%%%%%%%%%%%%%%%%%%%%%%%%%%%%
% Chargment du fichier de bibliographie
%
\bibliography{biblio}


%%%%%%%%%%%%%%%%%%%%%%%%%%%%%%%%%%%%%%%%%%%%%%%%%%%%%%%%%%
% Information à propos du travail de bachelor
%
% Pour utiliser une valeur dans le document, prefixer le nom
% pas @ ou @short. example : 
%    \title → \@title
%    \title → \@shorttitle
%
% \title[Short title]{TB title}
% Titre du travail de bachelo
\title[short title]{Titre du TB}

% \titlesuffix[suffix for short title]{suffix for TB title}
% Suffix au titre du TB (ex: "Résumé publiable" ou "Journal de travail")
% À ne définir que dans les documents où il est util
%\titlesuffix[suffix for short title]{suffix for TB title}

% \version[version other place]{verion on titlepage}
% Version du travail
\version[\inputMake{_dyn/git-version-short}]{\inputMake{_dyn/git-version}}

% \illustration[filename]
% Illustration de la page de titre
%  - Si pas de fichier préciser, la valeur est illustration
%  - Si la commande n'est pas appellé, il n'y a pas d'illustration
\illustration

% \illustrationheight[height]
% Hauteur de l'illustration (défaut: 23em)
%\illustrationheight[23em]

% \author[initial]{name}
% Nom et initial de l'étudiant
\author[NÉ]{Nom de l'étudiant}

% \authorfem or \authormasc
% Pour la langue française, le genre de l'étudiant·e pour les accords
% (You can use it with \if@authorgenderfem)
\authorfem
%\authormasc


% \school[acronym]{name}
% Nom et abréviation de l'école
\school[HEIG-VD]{Haute École d'Ingénierie et de Gestion du Canton de Vaud}

% \department[acronym]{name}
% Nom du département
\department[TIC]{Technologies de l'Information et de la Communication}

% \faculty[acronym]{name}
% Nom de la faculté
\faculty[TELE]{Télécommunications}

% \orientation[acronym]{name}
% Nom de l'orientation
\orientation[TS]{Sécurité de l'information}

% \bachelorof[acronym]{name}
% Type de travail de bachelor, utiliser dans le préambule
\bachelorof{Ingénierie} % Ingénierie ou Économie d'entreprise

% \confidential
% Indique que le TB est confidentiel
\confidential

% \proposedby{person's name}{Company name}{Company address}
% Entreprise ayant proposé le travail de bachelor
%  - S'il n'y a pas d'entreprise commenter la ligne
\proposedby{Prénom-nom}{Entreprise}{Adresse\\NPA Ville}

% \teacher[initial]{name}
% Nom et initial de l'enseignant·e
\teacher[PN]{P. nom}

% Gender of the teacher (for french version)
% (You can use it with \if@teachergenderfem)
\teacherfem
%\teachermasc

% \schoolyear{20xx--20xx}
% Academic year of bachelor
\schoolyear{2020--2021}

% Date du rendu (utiliser à diverse endroit)
\date{\today}

% Lieu de rendu (pour la signature)
\location{Yverdon-les-Bains}

